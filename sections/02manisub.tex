\chapter{r}
A integridade da infraestrutura offshore é uma questão-chave para a produção ininterrupta de petróleo e gás, bem como para a segurança dos funcionários embarcados. O envelhecimento das infraestruturas causada principalmente pela corrosão e fadiga gera um aumento da probabilidade de falha. Em casos extremos, a perda da integridade estrutural pode causar o colapso da estrutura resultando na perda completa da instalação e ocasionalmente em um desastre ecológico e humano.

O projeto de vida típico de uma plataforma é de 20 - 25 anos, no entanto, devido à demanda atual para a produção de petróleo e gás, muitas dessas plataformas continuarão em operaçãos além do projetado. A inspeção estrutural da região submersa destas estruturas apresenta um desafio logístico dado o meio subaquático. Logo, a importância do desenvolvimento de sistemas submarinos adequados para inspeções periódicas de integridade estrutural. Um agravante a este problema é o aumento do percentual da produção executada por equipamento submersos, que resulta em uma maior demanda de inspeção periódica dos equipamentos posicionados no fundo marinho.

Atualmente, as estruturas subaquáticas são inspecionados ou por ROVs ou mergulhadores. Ambos os métodos são dispendiosos e ineficientes, principalmente por duas razões:

\begin{itemize}
	\item necessidade de pessoal altamente especializado;
	\item e o tempo necessário para realizar inspeções manualmente.
\end{itemize}

Portanto, o principal desafio no problema de inspeção estrutural submarina reside no desenvolvimento de um método eficiente de inspeção, que não necessita de um pessoal altamente especializado e que pode ser utilizada diretamente a partir das plataformas, sem a necessidade de uma embarcação de suporte.

\section{Manipuladores subaquáticos}
\label{chap:sotamani}
Um manipulador, também conhecido com um braço robótico, é considerado a ferramenta mais adequada para executar operações de intervenção submarina. Assim, os veículos submarinos não tripulados (\textit{\acs{UUV}s}), como os veículos operados remotamente (ROVs) e, em alguns casos, os veículos subaquáticos autônomos (\textit{\acs{AUV}s}) são equipados com um ou mais manipuladores submarinos. UUVs com manipuladores são freqüentemente chamados de Manipulador de Veículos Subaquáticos (\textit{\acs{UVM}s}).

% Segundo \cite{sivvcev2018underwater}

% Algumas das tarefas que os manipuladores subaquáticos são projetados para executar incluem inspeção de tubos \cite{christ2013rov}, salvamento de objetos afundados \cite{chang2004distance}, limpeza de superfícies \cite{davey1999non}, abertura e fechamento de válvulas, perfuração, corte de cabos \cite{christ2013rov}, assentamento e reparo de cabos, limpeza de entulhos e redes de pesca, biológica \cite{jones2009using} e amostragem geológica \cite{noe2006surface}, etc. Em geral, os manipuladores estão localizados na parte frontal do veículo submerso, mas nem sempre é esse o caso, por exemplo, há veículos com um manipulador localizado na parte inferior lado \cite{ribas2011girona}.

\subsection{Estudo do estado da arte}
\label{sec:sotamani}
De forma abrangente o estado da arte do conhecimento sobre os sistemas manipuladores subaquáticos foi baseado inicialmente em um artigo fundamentado por

%\citeonline{sivvcev2018underwater}.
Os autores forneceram uma pesquisa sobre o uso da tecnologia de manipulação para uma variedade de operações de intervenção submarina e inspeção em diferentes áreas de aplicação offshore. Ambas as soluções de manipuladores subaquáticos comercialmente disponíveis e os sistemas de protótipos foram analisados. Tópicos relevantes foram discutidos, incluindo especificações técnicas de manipuladores, projeto mecânico, atuação, modelagem de robô (cinemática e dinâmica), abordagens de controle e algoritmos (controle de movimento, controle cinemático, planejamento de movimento) e uma comparação detalhada foi apresentada destacando vantagens e desvantagens de diferentes soluções presentes na tecnologia de manipulação submarina. Este tópico apresenta uma imagem atual da tecnologia existente a fim de fornecer uma fonte útil para futuras pesquisas no campo da robótica subaquática e manipulação. Fatores críticos que limitam o desempenho de manipuladores subaquáticos passaram a ser mais entendidos a partir da revisão abrangente deste estado da arte. Os autores recomendam fortemente que esses fatores sejam considerados durante o projeto de futuros sistemas manipuladores subaquáticos.
Além disso, não há manipuladores comerciais controláveis ​​por torque no momento. Portanto, muitos dos algoritmos de controle de baixo nível propostos não são aplicáveis ​​a sistemas comerciais e até mesmo à maioria dos protótipos. Pesquisas acadêmicas relativamente recentes que incluiam ensaios submarinos experimentais em ambiente de campo ou pelo menos em tanques de teste tem sido realizada na Ocean One, MARIS, TRIDENT, RAUVI, PANDORA, CManipulator e KORDI.
Os dois últimos tem concentrado suas pesquisas em sistemas de manipuladores de ROV hidráulicos comerciais, enquanto os restantes concentram nas intervenções de AUVs com manipuladores de protótipos elétricos.
Alguns dos projetos em curso na atualidade estão sendo desenvolvidos em ambientes relevantes de manipulação submarina, desta forma pode-se estabelecer uma pequena lista:
\begin{itemize}
	\item ROBUST \url{http://eu-robust.eu/};
	\item MERBOTS \url{http://www.irs.uji.es/project/merbots}; 
	\item DexROV \url{http://www.dexrov.eu/};
	\item Operations Support Engineering - MaREI \url{http://www.mmrrc.ul.ie/};
\end{itemize}